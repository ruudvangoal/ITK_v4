\documentclass{letter}
\usepackage{geometry,amsmath,amssymb}
\geometry{letterpaper}

%%%%%%%%%% Start TeXmacs macros
\newcommand{\subsection}[1]{\medskip\bigskip

\noindent\textbf{\Large #1}}
\newcommand{\tmop}[1]{\ensuremath{\operatorname{#1}}}
%%%%%%%%%% End TeXmacs macros

\begin{document}

\title{Parameter Scales Estimation}\author{PICSL, UPENN}\maketitle

\subsection{1. Parameter Scales Estimation}

We estimate the scales of transform parameters from their impacts on
deformation. Let us denote the parameters by $p = (p_1, p_2 \ldots p_n)$, and
the transform function by $T (x)$. The impact $s_i$ from a unit change of
parameter $p_i$ may be defined in multiple ways, such as the maximum shift of
voxels or the average norm of transform Jacobians.

The gradient descent step in $p$ is $\vartriangle p = \frac{\partial
T}{\partial p}$. If we rescale $p$ to $q = s \cdot p$ where $s$ is a diagonal
matrix $\tmop{diag} (s_1, s_2 \ldots s_n)$, a unit change of $q_i$ will have
the same impact on deformation for $i = 1... n$. Now the gradient descent step
in $q$ becomes $\vartriangle q = \frac{\partial T}{\partial q} =
\frac{\partial T}{\partial p} \cdot s^{- 1}$. And in space of $p$, the step
becomes $\vartriangle p' = \vartriangle q \cdot s^{- 1} = \frac{\partial
T}{\partial p} \cdot s^{- 2}$. That is, we use the squared reciprocal of the
deformation impacts from parameters as the gradient scales.

\subsection{2. Step Size Estimation}

Since the descrete image gradient may become invalid beyond one voxel, it is a
good practice to limit a deformation step to one voxel spacing[1]. We allow
users to specify a maximum displacement to normalize a gradient step.



[1] M. Jenkinson and S.M. Smith, "A global optimisation method for robust
affine registration of brain images," Medical Image Analysis, vol. 5(2), pp.
143-156, June 2001.



\end{document}
